  
\documentclass[11pt,a4paper]{ivoa}
\input tthdefs

\title{Mapping VO-DMl}

% see ivoatexDoc for what group names to use here
\ivoagroup{DM}

\author[????URL????]{Omar Laurino}
\author{Gerard Lemson}

\editor{Omar Laurino}
\editor{Gerard Lemson}

% \previousversion[????URL????]{????Concise Document Label????}
\previousversion{This is the first public release}
       

\begin{document}
\begin{abstract}
???? Abstract ????
\end{abstract}


\section*{Acknowledgments}

\section*{History of this document}\label{history-of-this-document}
\addcontentsline{toc}{subsection}{History of this document}

\textbf{TODO} migrate document's history

\section{Introduction}\label{introduction}

Data providers put a lot of effort in organizing and maintaining
metadata that precisely describes their data files. This information is
invaluable for users and for software developers that provide users with
user-friendly VO-enabled applications. For example, such metadata can
characterize the different axes of the reference system in which the
data is expressed, or the history of a measurement, like the publication
where the measurement was drawn from, the calibration type, and so
forth. In order to be interoperable, this metadata must refer to some
Data Model that is known to all parties: the IVOA defines and maintains
such standardized Data Models that describe astronomical data in an
abstract, interoperable way.

In order to enable such interoperable, extensible, portable annotation
of data files, one needs:

\begin{itemize}
\item
  A language to unambiguously and efficiently describe Data Models and
  their elements' identifiers (VO-DML, {[}@vodml{]}).
\item
  Pointers linking a specific piece of information (data or metadata) to
  the Data Model element it represents\footnote{This used to be the
    assumed role of the \texttt{@utype} attribute in VOTable and for
    example TAP. This document is based on the new \texttt{VODML}
    element in \texttt{VOTable} 1.4 {[}@votable{]}.}.
\item
  A mapping specification that unambiguously describes the mapping
  strategies that lead to faithful representations of Data Model
  instances in a specific format.
\end{itemize}

Without a consistent language for describing Data Models there can be no
interoperability among them, through reuse of models by models, or in
their use in other specifications. Such a language must be expressive
and formal enough to enable the serialization of data types of growing
complexity and the development of reusable, extensible software
components and libraries that can make the technological uptake of the
VO standards seamless and scalable.

For serializations to non-standard representations one needs to map the
abstract Data Model to a particular format meta-model. For instance, the
VOTable format defines \texttt{RESOURCE}s, \texttt{TABLE}s,
\texttt{PARAM}s, \texttt{FIELD}s, and so forth, and provides explicit
attributes such as \texttt{units}, \texttt{UCD}s, and \texttt{utypes}:
in order to represent instances of a Data Model, one needs to define an
unambiguous mapping between these meta-model elements and the Data Model
language, so to make it possible for software to be able to parse a file
according to its Data Model and to Data Providers to mark up their data
products.

While one might argue that a standard for portable, interoperable Data
Model representation would have been required before one could think
about such a mapping, we are specifying it only at a later stage. In
particular several different interpretations of \texttt{UTYPE}s have
been proposed and used {[}@usages{]}. This specification aims to resolve
this ambiguity.

As a matter of fact, existing files and services can be made compliant
according to this specification by simply \emph{adding} annotations and
keeping the old ones. So they do not need to \emph{change} them in such
a way that would necessarily make them incompatible with existing
software.

Several sections of this document are utterly informative: in
particular, the appendices provide more information about the impact of
this specification to the current and future IVOA practices.

This specification describes how to represent Data Model instances using
the VOTable schema. This representation uses the
\texttt{\textless{}VODML\textgreater{}} element introduced for this
purpose in VOTable v1.4 {[}@votable{]} and the structure of the VOTable
meta-model elements to indicate how instances of data models are stored
in VOTable documents. We show many examples and give a complete listing
of allowed mapping patterns.

In sections {[}-@sec:usecases{]} to {[}-@sec:info{]} we give an
introduction to why and how the VODML elements can be used to hold
pointers into the data models.

Section {[}-@sec:normative{]} is a rigorous listing of all valid
annotations, and the normative part of the specification.

The appendices contain additional material. {[}Section @sec:schema{]}
describes the VODML annotation element that was added to the VOTable
schema to support this mapping specification. {[}Section @sec:clients{]}
describes different types of client software and how they could deal
with VOTables annotated according to the current specification.

\textbf{Throughout the document we will refer to some real or example
Data Models. Please remember that such models have been designed to be
fairly simple, yet complex enough to illustrate all the possible
constructs that this specification covers. They are not to be intended
as actual DMs, nor, by any means, this specification suggests their
adoption by the IVOA or by users and or Data Providers. In some cases we
refer to actual DMs in order to provide an idea of how this
specification relates to real life cases involving actual DMs.}

\section{Use Cases}\label{sec:usecases}

\subsection{General Remarks}\label{general-remarks}

This specification provides a standardized mechanism for annotating
VOTables according to data models. Thus, it enables:

\begin{itemize}
\itemsep1pt\parskip0pt\parsep0pt
\item
  Data providers to annotate VOTables so to faithfully map VOTable
  contents to one or more data models, as long as such Data Models are
  expressed according to the VO-DML standard \citet{2018ivoa.spec.0910L}. In other
  terms, they can \emph{serialize} data model \emph{instances} in a
  standard, interoperable way. Some examples are provided below as
  concrete use cases.
\item
  Service clients to faithfully reconstruct the semantics of the data
  model instances they consume in VOTable format. Some concrete examples
  are also provided below.
\end{itemize}

One of the main goals of this specification is also to alleviate data
modelers from the burden of defining special serialization strategies
for their data models, at least in most cases. Specialized
serializations might be defined if there are special constraints in term
of efficiency or effectiveness which require specialized serialization
schemes.

As a corollary to the above paragraph, client applications can also be
implemented on top of standardized Input/Output libraries that implement
the present specification, as the serialization mechanisms are
standardized across data models. Without this specification clients
would need to be coded against specialized serializations for each data
model. Similarly, data providers can now serialize instances of any data
model to VOTable using the same annotation mechanism.

As a result, this mapping specification should enable a large number of
concrete use cases to be implemented, by reducing the annotation burden
on both data providers and data consumers, improving the overall
interoperability, at least for what VOTable is concerned.

This document also represents a template for mapping data model
instances to other formats (see \ref{sec:other-formats}).

\subsection{Concrete Use Cases}\label{concrete-use-cases}

\subsubsection{STC clients}\label{stc-clients}

A typical usage scenario may be a VOTable client that is sensitive to
certain models only, say Space Time Coordinates (STC). Such a tool may
be written to understand annotations for instance of STC types,
manipulate such instances, and write them back to disk.

By finding an element mapped to a type definition from the STC model,
the application may infer for example that it represents a coordinate on
the sky and use this information according to its requirements. For
example, the client may convert all positions expressed in a certain
coordinate frame to a different coordinate frame through some specific
transform.

Note that the client may never parse any VO-DML description files, as
the knowledge about the data model may be assumed when the client code
is developed.

Also, the STC annotations are the same in all the contexts in which an
STC type is used. So, for instance, if a VOTable describes catalogues of
sources, and each source has a position attribute describing its
coordinates in a specific reference frame, the \emph{instance}
describing the source's position would be annotated in the exact same
way as, say, the central position of a cone search query.

So, the STC tool may not necessarily understand other models where STC
types are \emph{used}, but it may still be able to find the instances of
those types.

Note that the same file may contain multiple tables and in any case STC
model instances defined in multiple coordinate systems or frames.

\subsubsection{VO-enabled plotting and fitting
applications}\label{vo-enabled-plotting-and-fitting-applications}

An application whose main requirement is to display, plot, and/or fit
data cannot be required to be aware of \emph{all} astronomical data
models. However, if these data models shared some common representation
of quantities, their errors, and their units, the application may
discover these pieces of information and structure a plot, or perform a
fit, with minimal user input: each point may be associated with an error
bar, upper/lower limits, and other metadata. The application remains
mostly model-agnostic: it is not required to \emph{understand}
high-level concepts like Spectrum, or Photometry.

Also, knowledge about the basic building-block types like quantities and
coordinates, may be hard-coded during development.

\subsubsection{Data discovery portals}\label{data-discovery-portals}

Data discovery portals allow users to query VO services, display
metadata, filter responses, and fetch datasets.

While these applications may not be particularly interested in specific
data models, standardized annotation mechanisms may allow their
developers to dynamically capture the structure of the metadata, and
provide better exploratory tools rather than flat tables.

Consider for example filtering tables according to an arbitrary physical
quantity, say for instance the spectral coverage of a spectrum, the
filter with which an image was taken, or the luminosity of a source in a
certain band. The portal may provide a friendly interface that allows
users to select the physical quantities using standardized
representations. It may do so dynamically for all the pieces of metadata
present in the dataset, rather than limiting this functionality to a set
of hard-coded metadata properties.

Also, the application may group data and metadata in a tree of concepts
the user is familiar with, rather than flattening the instrinsic
structure of the data.

By allowing such applications to faithfully represent data model
instances the way they were curated and annotated by data providers, and
according to the scientific domain with which users are familiar, this
specification may enable users to easily make sense of the file
contents, even if the application lacks any knowledge of high-level data
models.

Parsing VO-DML description files may be useful to provide the user with
even more, and more accurate, information.

\subsubsection{Color-color diagrams}\label{color-color-diagrams}

The creation of a color-color diagram requires knowledge of the
semantics of the rows and columns in a table, e.g.~a source catalog.
Also, some columns may be grouped together in a structured way, e.g.~a
luminosity measurement that goes with its error and a reference to the
photometry filter it is associated with, although such columns might not
be adjacent in the file.

Usually, plotting applications are not aware of the semantics of the
columns in a table, and users may have to select all the relevant
columns in order to produce a meaningful plot. They may also have to
convert between units, and so on.

With a standardized annotation and knowledge of the annotations for
basic quantities a plotting application may find that there are
luminosity measurements in a VOTable and allow users to display
color-color diagrams, with error bars, and possibly with domain-specific
actions like unit conversions, seamlessly and requiring minimal input.

There are many examples of very specific use cases that may be
implemented by science applications that are aware of the semantics of
specific models and their annotations.

\subsubsection{Validators}\label{validators}

The existence of an explicit data model representation language and of a
precise, unambiguous mapping specification enables the creation of
universal validators, just as it happens for XML and XSD: the validator
may parse the data model descriptions imported by the VOTable and check
that the file represents valid instances of one or more data models.

\subsubsection{VO Publishing Helper}\label{vo-publishing-helper}

There is some complexity involved in understanding how to publish data
in the Virtual Observatory. The availability of a standard for
serializing data model instances can provide tools for publishers to
build templates of their responses.

Such application may help data providers in interactively mapping Data
Models elements to their files or DB tables, either producing a VOTable
template with the appropriate annotations, or by creating a DAL service
on the fly.

The application is not required to be model-aware, since it may get all
the information from the standardized description files and the mapping
specification.

\subsubsection{VO Importer}\label{vo-importer}

Users and Data Providers may have non-compliant files that they want to
convert to a VO-compliant format according to some data models: a
generic, model-unaware importer application may allow them to do so for
any standard data model with a proper description file.

\subsection{Generic Use Cases}\label{generic-use-cases}

This section generalizes the previous one by stating the same use cases
in a more abstract, generic formulation. It also adds some more generic
use cases that this specification addresses or enables.

\subsubsection{Serialize and de-serialize instances according to a data
model}\label{serialize-and-de-serialize-instances-according-to-a-data-model}

Data providers may want to serialize data and metadata in VOTable
according to a specific data model with a standardized description.

A client may build an in-memory faithful representation of that instance
according to the data provider's annotations, assuming the knowledge of
a finite set of data model identifiers, of a full data model, or by
parsing standard VO-DML data model specifications.

\subsubsection{Annotate files according to multiple
models}\label{annotate-files-according-to-multiple-models}

Data providers may find it useful to annotate a file according to
different data models for different classes of data consumers. Also,
they may decide to provide annotations according to different versions
of a specific data model, to favor backward compatibility with older
clients.

\subsubsection{Representing cross matches and linking files
together}\label{representing-cross-matches-and-linking-files-together}

It is often the case that two or more files or tables represent
different pieces of information regarding the same astronomical sources
of objects. In these cases, one or more columns usually are used as keys
to identify instances in such tables.

For instance, the output of a cross-matching service may provide the IDs
of the cross matched sources along with some data regarding the
cross-matching process, while most of the data about the sources
themselves may be stored in different tables.

This is a very common relational pattern. A standardized annotation with
Object-Relational Mapping capabilities may be used to connect different
tables and provide users with a unified view according to data models
the user is familiar with.

It is then possible to link different views of the datasets, with an
additional layer of semantics. For instance, an application may display
the image of a region of the sky, and a catalogue may contain
information about sources in the catalogue.

A user may want to link the image to the catalogue. When no a-priori
link between the image coordinates and the positions in the catalogue is
known, users need to set such links themselves, by selecting the
relevant columns. With standardized annotations according to specific
data models applications can figure out the links themselves, and ask
the user to intervene only when there is ambiguity or lack of
information, or when the user wants to make a custom link.

Similarly, one may produce a color-color diagram from magnitudes stored
in different tables as long as there is a known mapping from source
identifiers among tables, and a standardized annotation of all the
tables involved.

\subsubsection{Mission-specific data model
extensions}\label{mission-specific-data-model-extensions}

One may identify data and metadata features that are common to a certain
astronomical domain, e.g.~catalogues of astronomical sources. These are
the models the IVOA sanctions in standards.

However, different missions, or archives, or applications will most
certainly have specific additional features that are not captured by
such common, standardized data models.

One may express such extensions and their instances in such a way that:
* data providers may easily annotate specialized instances, including
the additional information, in a standardized, interoperable fashion. *
clients of the common, standard data models may still find instances of
these parent models in files serializing specialized instanced. * a
model annotation that is valid according to a specialized data model is
also valid according to the parent model.

This use case can be formalized in terms of inheritance and
polymorphism.

Inheritance allows models to specialize types defined in other data
models. Polymorphism is the common object-oriented design concept that
says that the value of a property may be an instance of a \emph{subtype}
of the declared type.

Typed languages such as Java support a casting operation, which provides
more information to the interpreter about the type it may expect a
certain instance to be.

A client must be able to identify the information about a standard type,
even if the serialization includes instances of its subtypes. Similarly,
a client should have enough information to \emph{cast} an instance from
the declared type to the actual subtype.

\subsection{Growing complexity: simple, advanced, and guru
clients}\label{sec:clients}

According to the use cases depicted above, we can classify clients in
terms of how they parse the VOTable in order to harvest its content. Of
course, in the real word such distinction is somewhat fuzzy, but this
section tries and describe the different levels of usage of this
specification.

This classification is useful because it shows how implementations can
be based on different assumptions. Some clients can focus on few
hard-coded elements, while other clients can dynamically address complex
tasks.

\subsubsection{Simple clients}\label{simple-clients}

We say that a client is \emph{simple} if:

\begin{itemize}
\item
  it does not parse the VO-DML description file
\item
  it assumes the a priori knowledge of one or more data models
\item
  it discovers information by looking for a set of predefined vodml-refs
  in the VOTable
\end{itemize}

In other terms, a simple client has knowledge of the data model it is
sensitive to, and simply discovers information useful to its own use
cases by traversing the \texttt{VODML} element.

Examples of such clients are the DAL service clients that allow users to
discover and fetch datasets. They may just inspect the response of a
service and present the user with a subset of its metadata. They do not
\emph{reason} on the content, and they are not interested in the
structure of the serialized objects.

If such clients allow users to download the files that they load into
memory, they should make sure to preserve the structure of the metadata,
so to be interoperable with other applications that might ingest the
same file at a later stage.

\subsubsection{Advanced clients}\label{advanced-clients}

We say that a client is \emph{advanced} if:

\begin{itemize}
\item
  it does not parse the VO-DML description file
\item
  it is interested in the structure of the serialized instances
\item
  can parse the elements defined in this specification
\end{itemize}

Examples of such clients are discovery portals or science applications
that display information to the user in a structured way, e.g.~by
plotting it, or by displaying its metadata in a user-friendly format.
Possibly, such applications may allow users to save versions of the
serialization after it has been manipulated.

Such clients may not assume any knowledge of any specific data models.
In some cases they may assume knowledge of some types from some basic,
common data models, to perform additional tasks.

Even if such applications may be model-unaware, they may allow users to
build Boolean filters on a table, using a user-friendly tree
representing the whole metadata. This exposes all the metadata provided
by the Data Provider in a way that may not be meaningful for the
application, but that may be meaningful for the user.

\subsubsection{Guru clients}\label{guru-clients}

We say that a client falls into this category if:

\begin{itemize}
\item
  it parses the VO-DML descriptions
\item
  it does not assume any a priori knowledge of any data models.
\end{itemize}

Such applications can, for example, dynamically allow users and data
providers to map their files or databases to the IVOA data models in
order to make them compliant, or display the content of any file
annotated according to this standard.

This specification allows the creation of universal validators
equivalent to the XML/XSD ones.

It also allows the creation of VO-enabled frameworks and reusable
libraries. For instance, a Python universal I/O library can parse any
VOTable according to the data models it uses, and dynamically build
structured objects on the fly, so that users can directly interact with
those objects or use them in their scripts or in science applications,
and then save the results in a VO-compliant format.

Java and Python guru clients could automatically generate interfaces for
representing data models and dynamically implement those interfaces at
runtime, maybe building different views of the same file in different
contexts.

Notice that guru frameworks and libraries can be used to build advanced
or even simple applications in a user-friendly way, abstracting the
developers from the technical details of the standards and using
scientific concepts as first class citizens instead.

\subsection{Formats other than VOTable}\label{sec:other-formats}

We want to explicitly note that this specification covers the VOTable
format only.

Other mapping specifications can and will provide standardized
strategies for mapping Data Models to formats other than VOTable.

Part of the implementation efforts related to the present specification
was to validate the standard against prototype serializations in JSON
and YAML formats.

Mapping specifications targeting additional formats can use this
document as a template.

\section{The need for a mapping
language}\label{the-need-for-a-mapping-language}

When encountering a data container, i.e.~a file or database containing
data, one may wish to interpret its contents according to some external,
predefined data model. That is, one may want to try to identify and
extract instances of the data model from amongst the information. For
example in the \emph{global as view} approach to information
integration, one identifies elements (e.g.~tables) defined in a global
schema with views defined on the distributed databases\footnote{See, for
  example,
  http://logic.stanford.edu/dataintegration/chapters/chap01.html}.

If one is told that the data container is structured according to some
standard serialization format of the data model, one is done. I.e. if
the local database is an exact \emph{implementation} of the global
schema, one needs no special annotation mechanism to identify these
instances. An example of this is an XML document conforming to an XML
schema that is an exact physical \emph{representation} of the data
model. We call such representations \emph{faithful}.

But in an information integration project like the IVOA, which aims to
homogenize access to many distributed heterogeneous data sets, databases
and documents are in general \emph{not} structured according to a
standard representation of some predefined, global data model. The best
one may hope for is to obtain an \emph{interpretation} of the data set,
defining it as a \emph{custom serialization} of the result of a
\emph{transformation} of the global data model\footnote{Or alternatively
  as a transformation of a (standard) serialization of the data model.}.
For example, even if databases themselves are exact replications of a
global data model, results of general queries will be such custom
serializations.

To interpret such a custom serialization one generally needs extra
information that can provide a \emph{mapping} of the serialization to
the original model. If the serialization \emph{format} is known, this
mapping may be given in phrases containing elements both from the
serialization format and the data model. For example if our
serialization contains data stored in `rows' in one or more `tables'
that each have a unique `name' and contain `columns' also with a `name',
you might be able to say things like:

\begin{itemize}
\item
  The rows in this table named SOURCE contain \emph{instances} of
  \emph{object type} `Source' as defined in \emph{data model}
  `SourceDM'.
\item
  The \emph{type}'s `name' \emph{attribute} (having \emph{datatype}
  `string', a \emph{primitive type}) also acts as the \emph{identifier}
  of the Source \emph{instances} and is stored in the single column with
  ID `name'.
\item
  The \emph{type's} `classification' \emph{attribute} is stored in the
  table column CLASSIFICATION (from the \emph{data model} we know its
  \emph{datatype} is an \emph{enumeration} with certain \emph{values},
  e.g. `star', `galaxy', `agn').
\item
  The \emph{type's} `position' \emph{attribute} (being of
  \emph{structured data type} `SkyCoordinate' defined in \emph{model}
  `SourceDM') is stored over the two columns RA and DEC, where RA stores
  the SkyCoordinate's \emph{attribute} `longitude', DEC stores the
  `latitude` \emph{attribute}. Both must be interpreted using an
  \emph{instance} of the SkyCoordinateSystem \emph{type}, This
  \emph{instance} is stored in 1) another document elsewhere, referenced
  by a \emph{reference} to a URI, or 2) in this document, by means of an
  \emph{identifier.}
\item
  \emph{Instances} from the \emph{collection} of luminosities of the
  Source \emph{instances} are stored in the same row as the source
  itself. Columns MAG\_U and ERR\_U give the `magnitude' and `error'
  \emph{attributes} of \emph{type} LuminosityMeasurement in the ``u
  band'', an \emph{instance} of the Filter \emph{type}. (stored
  elsewhere in this document (`a \emph{reference} to this Filter
  instance is \ldots{}'). Columns MAG\_G and ERR\_G \ldots{} etc.
\item
  Luminosity \emph{instances} also have a filter \emph{relation} that
  points to instances of the PhotometryFilter \emph{structured data
  type}, defined in the IVOA PhotDM model, whose \emph{package} is
  imported by the SourceDM.
\end{itemize}

In this example the \emph{emphasized} words refer to concepts defined in
VO-DML, a meta-model that is used as a formal language for expressing
data models. The use of such a modeling language lies in the fact that
it provides formal, simple, and implementation neutral definitions of
the possible structure, the `type' and `role' of the elements from the
actual data models that one may encounter in the serialization
(SourceDM). This can be used to constrain or validate the serialization,
but more importantly it allows us to formulate mapping rules between the
serialization format (itself a kind of meta-model) and the meta-model,
independent of the particular data models used; for example rules like:

\begin{itemize}
\item
  An \emph{object type} MUST be stored in an `INSTANCE'.
\item
  A `\emph{primitive type}' MUST be stored in a `COLUMN'.
\item
  A \emph{reference} MUST identify an \emph{object type} \emph{instance}
  represented elsewhere, possibly in another `table', possibly in the
  same table, possibly in another document.
\item
  An \emph{attribute} SHOULD be stored in the same table as its
  containing \emph{object type}.
\item
  etc
\end{itemize}

Clearly free-form English sentences as the ones in the example are not
what we are after. If we want to be able to identify how a data model is
represented in some custom serialization we need a formal, computer
readable mapping language.

One part of the mapping language should be anchored in a formally
defined serialization language. After all, for some tool to interpret a
serialization, it MUST understand its format. A completely freeform
serialization is not under consideration here. This document assumes the
target meta-model is VOTable.

The mapping language must support the interpretation of elements from
the serialization language in terms of elements from the data model. If
we want to define a generic mapping mechanism, one by which we can
describe how a general data model is serialized inside a VOTable, it is
necessary to use a general data model \emph{language} as the target for
the mapping, such as the one described above. This language can give
formal and more explicit meaning to data modeling concepts, possibly
independent of specific languages representation languages such as XML
schema, Java, or the relational model.

This document uses VO-DML as the target language.

The final ingredient in the mapping language is a mechanism that ties
the components from the two different meta-models together into
\emph{sentences}. This generally requires some kind of explicit
annotation, some meta-data elements that provide an identification of
source to target structure. This document uses an extension to VOTable
with a VODML element which can provide this link in a rather simple
manner.

This solution is sufficient and it is in some sense the simplest and
most explicit approach for annotating a VOTable. It may \emph{not} be
the most natural or suitable approach for other meta-models such as FITS
or TAP\_SCHEMA. We discuss this at the end of this document.

\section{Mapping with the \texttt{VODML}
element.}\label{mapping-with-the-vodml-element.}

This section summarizes the technical basis of this reccomendation.

The present specification, in conjunction with with VO-DML
recommendation \cite{2018ivoa.spec.0910L} provides a formal mechanism for using such
data model identifiers, although different from the original
\texttt{@utype} attribute definition and its usages \cite{note:utypeusage}.

VOTable 1.2 introduced the \texttt{@utype} attribute, which was intended
to represent ``pointers into a data model''. A precise and formal
definition on how this \emph{pointing} was to be achieved and a
description of its meaning was missing though.

First, a formal definition of the target of the pointers was missing. To
solve this, data models were usually accompanied by a list of
\emph{utypes} (\textbf{TODO} TBD refer to STC, Characterization,
Spectrum?), and these could be used as values for said
\texttt{@utype}, be it in VOTable or for example in the Table Access
Protocol metadata. These were not linked in any formal, machine readable
way to the underlying data model.

Basically it means that the data model is represented solely by a list
of attributes, which does not do justice to the complexity of data
models describing complex data products like Data Cubes or the
provenance of Simulations. These contain complex object hierarchies
organized in graphs with various types of relations between individual
objects. It also proved difficult to express the relationship among
different, but overlapping, data models, with much discussion centred on
the question how to reuse utypes from one model in the definition of
another.

The approach is basically not much more than another vocabulary, similar
to UCDs \cite{2018ivoa.spec.0527M}, or SKOS vocabularies \cite{2009ivoa.spec.1007G}, obtained by
different means. Efforts were made to provide some structure to these
values that might provide some hints of their location in a model, but
there was no formal mechanism on how to derive that structure and it was
unclear whether it could truly represent the richness of the existing
and future data models. In particular there was no standard defined how
this could be achieved and no common usage patterns were discovered
\cite{note:utypeusage}.

VO-DML provides a formal target for these pointers in the data model
itself and formally defines how models can be reused in the definition
of other, dependent models. Precisely \emph{how} to use these pointers
in a VOTable to provide a complete annotation useful for
interoperability requires more work though.

The current specification provides such a definition. It shows how data
publishers can identify also the more complex data model elements such
as structured types and relationships inside some published data source,
be it a VOTable or relational database published through the TAP
protocol.

This specification defines various \emph{mapping patterns} from VOTable
to VO-DML. Such a pattern identifies a VOTable element with a VO-DML
element. The VO-DML element is said to be \emph{represented} by the
VOTable element. The mapping pattern indicates that instances of
identified VO-DML types are present in the VOTable. These may be atomic
\emph{values} (instances of VO-DML ValueTypes {[}@vodml{]}) or
represented by cells in a table column identified by a \texttt{FIELD}.
Alternatively they may be instances of structured types.

The extension to the VOTable schema is reproduced in {[}@sec:schema{]}.

\subsection{VODMLReference}\label{vodmlreference}

This XML type represents a reference to a single element in a VO-DML/XML
document. It take over the role of the \texttt{@utype} attribute in this
regards. Whenever we wish to refer to instances of the VODMLReference
type we will call them \textbf{vodmlref}-s. A vodmlref is a string with
the following syntax:

\begin{verbatim}
vodmlref ::= prefix ‘:’ vodml-id
\end{verbatim}

The prefix identifies the model in which the element identified by the
suffix is defined.

\textbf{vodml-ids} are always considered opaque, meaning that clients
have no reason to parse them. They are identifiers mapping VOTable
elements to VO-DML elements in the identified data model. Thus, they
must follow the same syntax rules defined in the VO-DML/Schema document.

Prefixes MUST be exactly the same as the \textbf{name} attribute of the
model in the VO-DML/XML document that defines it. They are sequences of
{[}A-Za-z0-9\_-{]}, and they are case sensitive.

For new models, that are not (yet) standardized or for custom data
models used in a smaller community, it is recommended to form DM
prefixes as
\texttt{\textless{}author-acronym\textgreater{}\_\textless{}dm-name\textgreater{}},
where the \texttt{\textless{}dm-name\textgreater{}} is the name of a
standard data model; thus, NED's derivation of spec could have
\texttt{ned\_spec} as a prefix, CDS's derivation \texttt{cds\_spec}.

Prefixes correspond to major versions for the corresponding data models.
Thus, \textbf{vodmlrefs} remain constant over ``compatible'' changes in
the sense of {[}@vodml{]}.~In consequence, clients must assume a
compatible extension when encountering an unknown \textbf{vodmlref} with
a known prefix (and should in general not fail).

Another consequence of this rule is that there may be several VO-DML
URLs for a given prefix. ~To identify a data model, use the prefix, not
the VO-DML URL, which is intended for retrieval of the data model
definition exclusively. ~In case a client requires the exact minor
version of the data model, it must inspect the models declarations as
described in {[}@sec:normative{]}

(\textbf{TODO} OL: This doesn't feel right. I believe minor versions
should be uniquely identified by a URI and without having to parse the
descriptor, especially since we have started talking about registering
models in the Registry.)

\paragraph{How to look for a vodmlref in a
document}\label{how-to-look-for-a-vodmlref-in-a-document}

(\textbf{TODO} to fill out once the syntax has settled)

\section{General information about this spec}\label{sec:info}

\subsection{Sample model and
instances}\label{sample-model-and-instances}

(\textbf{TODO} This needs to be filled with the designated sample model)

\subsection{Single-table representations and Object-Relational
Mapping}\label{single-table-representations-and-object-relational-mapping}

Broadly speaking, this specification is all about Object-Relational
Mapping (ORM). Data Models are represented in VO-DML according to an
Entity-Relationship paradigm, in a fashion that is implementable by
relational databases, object oriented languages, and possibly to certain
document oriented dabases as well.

As VOTable can represent several tables in the same file with rich
metadata, one can look at VOTable as a database that can represent
complex relational models.

Such models are usually defined in terms of entities, with each table
representing each entity, and relationships that can be expressed as
tables themselves or as constraints on the values in the tables, and
most often with a combination of tables and constraints. For instance, a
Many-To-Many relationship between two Entities is usually represented in
the relational model as a table holding IDs of instances from the tables
representing the Entities, with Foreign Keys constraints.

Astronomers mainly work with single tables that hold flattened
representations of relatively simple models, although in some cases
complex data models are serialized in several tables inside the same
file.

This specification covers both requirements. Serializations of simple
models in a flattened table are easier to achieve than complex ORM
mappings where information is normalized into different tables, but they
are both achievable in VOTable. Moreover, the hybrid case of partly
de-normalized representations, where the model is only partly
normalized, is more challenging but should also be addressable in terms
of this specification.

In any case, the examples in {[}@sec:normative{]} are focused on the
single-table, flattened representations of instances according to some
data model. Some of the patterns described in these sections are also
applicable to simple ORM cases. Especially the sections dealing with
mapping reference and composition relations also deal with the more
complex cases of proper ORM mappings, where data is partly or completely
normalized into different tables.

The simple and complex ORM patterns described by this specification
usually belong to very different concrete use cases, so it should be
acceptable in a broad range of cases that implementers, both on the
server and on the client side, focus on the single-table mappings. Data
providers requiring more complex patterns, more advanced applications,
or applications built on top of standard software libraries that
implement this specification as a whole will need to take advantage of
the ORM mapping patterns.

\section{Patterns for annotating VOTable
{[}NORMATIVE{]}}\label{sec:normative}

In this section we list all legal mapping patterns that can be used to
express how instances of VO-DML-defined types are represented in a
VOTable and the possible roles they play. It defines the VOTable
annotation syntax, what restrictions there are, and how to interpret the
annotation semantically.

The organization of the following sections is based on the XML types
introduced in the new VOTable schema. The entire VO-DML annotation is
included in a new \texttt{VODML} element and its descendents. The
following sections introduce each element and how it is used to map the
VO-DML concepts to VOTable documents.

In particular, {[}@sec:norm-vodml;@sec:norm-model;@sec:norm-relations{]}
describe the schema elements that provide data model annotations, while
{[}@sec:norm-types{]} inverts this approach and shows how to map VO-DML
concepts to the VOTable elements that have been introduced.

Some comments on how we refer to VOTable and VO-DML elements:

\begin{itemize}
\item
  When referring to VOTable elements we will use the notation by which
  these elements will occur in VOTable documents, i.e.~in general ``all
  caps'', E.g. \texttt{GROUP}, \texttt{FIELD}, (though
  \texttt{FIELDref}).
\item
  When referring to an XML attribute on a VOTable element we will prefix
  it with a `@', e.g. \texttt{@id}, \texttt{@ref}.
\item
  References to VO-DML elements will be CamelCase and in
  \textbf{\texttt{bold face}}, using their VO-DML/XSD type definitions.
  E.g. \textbf{\texttt{ObjectType}}, \textbf{\texttt{Attribute}}.
\end{itemize}

The following list defines some shorthand phrases (\emph{italicized}),
which we use in the descriptions below:

\begin{itemize}
\item
  Generally when using the phrase \emph{meta-type} we mean a ``kind of''
  type as defined in VO-DML. These are \textbf{\texttt{PrimitiveType}},
  \textbf{\texttt{Enumeration}}, \textbf{\texttt{DataType}} and
  \textbf{\texttt{ObjectType}}.
\item
  With \emph{atomic type} we will mean a \textbf{\texttt{PrimitiveType}}
  or an \textbf{\texttt{Enumeration}} as defined in VO-DML.
\item
  A \emph{structured type} will refer to an \textbf{\texttt{ObjectType}}
  or \textbf{\texttt{DataType}} as defined in VO-DML.
\item
  With a property \emph{available on} or \emph{defined on} a
  (structured) type we will mean an \textbf{\texttt{Attribute}} or
  \textbf{\texttt{Reference}}, or (in the case of
  \textbf{\texttt{ObjectType}}s) a \textbf{\texttt{Composition}} defined
  on that type itself, or inherited from one of its base class
  ancestors.
\item
  A VO-DML \textbf{\texttt{Type}} \emph{plays a role} in the definition
  of another (structured) type if the former is the declared data type
  of a property available on the latter.
\item
  When writing that a VOTable element \emph{represents} a certain VO-DML
  type, we mean that the VOTable element is mapped either directly to
  the type, or that it identifies a role played by the type in another
  type's definition.
\item
  A \emph{descendant} of a VOTable element is an element contained in
  that element, or in a descendant of that element. This is a standard
  recursive definition and can go up the hierarchy as well: an
  \emph{ancestor} of an element is the direct container of that element,
  or an ancestor of that container.
\end{itemize}

\textbf{Regarding the \emph{normative} aspects of this specification}

When we say this section is NORMATIVE we mean that:

\begin{enumerate}
\def\labelenumi{\arabic{enumi}.}
\itemsep1pt\parskip0pt\parsep0pt
\item
  when a client finds an annotation pattern conforming to one defined
  here, that client is justified in interpreting it as described in the
  comments for that pattern. It is an ANNOTATION ERROR if that were to
  lead to inconsistencies\footnote{E.g. when interpreting an
    \texttt{\textless{}INSTANCE\textgreater{}} as a certain
    \textbf{\texttt{ObjectType}}, if one of its children is not
    annotated or identifies a child element that is not available on the
    type, this is an error. For each pattern there is a set of rules
    that, if broken, are annotation errors. (\textbf{TODO} we better
    strive to make these comprehensive. OL thinks this is too strict of
    a default rule. We should probably be more lenient by default and
    add strictness where required. A lot is already mandated by the new
    schema.)}.
\item
  when a client encounters a pattern not in this list, the client SHOULD
  ignore it. Interpreting it as a mapping to a data model MAY work, but
  is not mandated and other clients need not conform to this.
\end{enumerate}

\subsection{The \texttt{VODML} element}\label{sec:norm-vodml}

\subsubsection{Example}\label{example}

\begin{verbatim}
file: instances/models.votable.xml
classes:
  - xml
caption: A Caption
id: lst:code
----
\end{verbatim}

\subsubsection{Schema Constraints}\label{schema-constraints}

\subsubsection{Models Declaration:
\texttt{MODELS}}\label{models-declaration-models}

\subsubsection{Global Instances:
\texttt{GLOBALS}}\label{global-instances-globals}

\subsubsection{Tabular Instances:
\texttt{TEMPLATES}}\label{tabular-instances-templates}

\subsubsection{Instance Annotation:
\texttt{INSTANCE}}\label{instance-annotation-instance}

\subsection{Model Declaration: MODEL}\label{sec:norm-model}

\subsection{Relations}\label{sec:norm-relations}

\subsubsection{Attributes: \texttt{INSTANCE}}\label{attributes-instance}

\subsubsection{Attributes: \texttt{LITERAL}}\label{attributes-literal}

\subsubsection{Attributes: \texttt{CONSTANT}}\label{attributes-constant}

\subsubsection{Template Attributes:
\texttt{COLUMN}}\label{template-attributes-column}

\subsubsection{Containers:
\texttt{CONTAINER}}\label{containers-container}

\subsubsection{Containers:
\texttt{PRIMARYKEY}}\label{containers-primarykey}

\subsubsection{Containers:
\texttt{FOREIGNKEY}}\label{containers-foreignkey}

\subsubsection{References:
\texttt{REFERENCE}}\label{references-reference}

\subsubsection{References: \texttt{IDREF}}\label{references-idref}

\subsubsection{References:
\texttt{REMOTEREFERENCE}}\label{references-remotereference}

\subsubsection{References:
\texttt{FOREIGNKEY}}\label{references-foreignkey}

\subsubsection{Compositions:
\texttt{COMPOSITION}}\label{compositions-composition}

\subsubsection{Compositions:
\texttt{INSTANCE}}\label{compositions-instance}

\subsubsection{Compositions:
\texttt{EXTINTANCES}}\label{compositions-extintances}

\subsection{Representing Types}\label{sec:norm-types}

\subsubsection{\textbf{\texttt{PrimitiveType}}}\label{primitivetype}

\subsubsection{\textbf{\texttt{Enumeration}} and
\textbf{\texttt{EnumerationLiteral}}}\label{enumeration-and-enumerationliteral}

\subsubsection{\textbf{\texttt{ObjectType}}}\label{objecttype}

\subsubsection{\textbf{\texttt{DataType}}}\label{datatype}

\subsubsection{Type Extensions}\label{type-extensions}

\subsubsection{Quantities}\label{quantities}

\section{The VODML annotation elements in the VOTable schema}\label{sec:schema}

\pagebreak
\bibliography{ivoatex/ivoabib,ivoatex/docrepo}

\end{document}